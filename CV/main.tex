\documentclass[10pt, letterpaper]{article}
\usepackage[utf8]{inputenc}
\usepackage[english, russian]{babel}

% Packages:
\usepackage[
    ignoreheadfoot, % set margins without considering header and footer
    top=0.5cm, % seperation between body and page edge from the top
    bottom=0.5 cm, % seperation between body and page edge from the bottom
    left=2 cm, % seperation between body and page edge from the left
    right=2 cm, % seperation between body and page edge from the right
    footskip=1.0 cm, % seperation between body and footer
    % showframe % for debugging 
]{geometry} % for adjusting page geometry
\usepackage{titlesec} % for customizing section titles
\usepackage{tabularx} % for making tables with fixed width columns
\usepackage{array} % tabularx requires this
\usepackage[dvipsnames]{xcolor} % for coloring text
\definecolor{primaryColor}{RGB}{0, 79, 144} % define primary color
\usepackage{enumitem} % for customizing lists
\usepackage{fontawesome5} % for using icons
\usepackage{amsmath} % for math
\usepackage[
    pdftitle={John Doe's CV},
    pdfauthor={John Doe},
    pdfcreator={LaTeX with RenderCV},
    colorlinks=true,
    urlcolor=primaryColor
]{hyperref} % for links, metadata and bookmarks
\usepackage[pscoord]{eso-pic} % for floating text on the page
\usepackage{calc} % for calculating lengths
\usepackage{bookmark} % for bookmarks
\usepackage{lastpage} % for getting the total number of pages
\usepackage{changepage} % for one column entries (adjustwidth environment)
\usepackage{paracol} % for two and three column entries
\usepackage{ifthen} % for conditional statements
\usepackage{needspace} % for avoiding page brake right after the section title
\usepackage{iftex} % check if engine is pdflatex, xetex or luatex

% Ensure that generate pdf is machine readable/ATS parsable:
\ifPDFTeX
    \input{glyphtounicode}
    \pdfgentounicode=1
    % \usepackage[T1]{fontenc} % this breaks sb2nov
    \usepackage[utf8]{inputenc}
    \usepackage{lmodern}
\fi



% Some settings:
\AtBeginEnvironment{adjustwidth}{\partopsep0pt} % remove space before adjustwidth environment
\pagestyle{empty} % no header or footer
\setcounter{secnumdepth}{0} % no section numbering
\setlength{\parindent}{0pt} % no indentation
\setlength{\topskip}{0pt} % no top skip
\setlength{\columnsep}{0cm} % set column seperation
\makeatletter
\let\ps@customFooterStyle\ps@plain % Copy the plain style to customFooterStyle
\makeatother
\pagestyle{empty} 

\titleformat{\section}{\needspace{4\baselineskip}\bfseries\large}{}{0pt}{}[\vspace{1pt}\titlerule]

\titlespacing{\section}{
    % left space:
    -1pt
}{
    % top space:
    0.3 cm
}{
    % bottom space:
    0.3 cm
} % section title spacing

\renewcommand\labelitemi{$\circ$} % custom bullet points
\newenvironment{highlights}{
    \begin{itemize}[
        topsep=0.10 cm,
        parsep=0.10 cm,
        partopsep=0pt,
        itemsep=0pt,
        leftmargin=0.4 cm + 10pt
    ]
}{
    \end{itemize}
} % new environment for highlights

\newenvironment{highlightsforbulletentries}{
    \begin{itemize}[
        topsep=0.10 cm,
        parsep=0.10 cm,
        partopsep=0pt,
        itemsep=0pt,
        leftmargin=10pt
    ]
}{
    \end{itemize}
} % new environment for highlights for bullet entries


\newenvironment{onecolentry}{
    \begin{adjustwidth}{
        0.2 cm + 0.00001 cm
    }{
        0.2 cm + 0.00001 cm
    }
}{
    \end{adjustwidth}
} % new environment for one column entries

\newenvironment{twocolentryED}[2][]{
    \onecolentry
    \def\secondColumn{#2}
    \setcolumnwidth{\fill, 4 cm}
    \begin{paracol}{2}
}{
    \switchcolumn \raggedleft \secondColumn
    \end{paracol}
    \endonecolentry
} % new environment for two column entries


\newenvironment{twocolentry}[2][]{
    \onecolentry
    \def\secondColumn{#2}
    \setcolumnwidth{\fill, 5 cm}
    \begin{paracol}{2}
}{
    \switchcolumn \raggedleft \secondColumn
    \end{paracol}
    \endonecolentry
} % new environment for two column entries

\newenvironment{header}{
    \setlength{\topsep}{0pt}\par\kern\topsep\centering\linespread{1.5}
}{
    \par\kern\topsep
} % new environment for the header

% save the original href command in a new command:
\let\hrefWithoutArrow\href

% new command for external links:
\renewcommand{\href}[2]{\hrefWithoutArrow{#1}{\ifthenelse{\equal{#2}{}}{ }{#2 }\raisebox{.15ex}{\footnotesize \faExternalLink*}}}

\makeatletter
\newcommand\notsotiny{\@setfontsize\notsotiny{9}{9}}
\makeatother
% \usepackage{tempora}
\renewcommand{\rmdefault}{cmr} % Шрифт с засечками
\renewcommand{\sfdefault}{cmss} % Шрифт без засечек
\renewcommand{\ttdefault}{cmtt} % Моноширинный шрифт
\begin{document}
    \notsotiny
    
    \newcommand{\AND}{\unskip
        \cleaders\copy\ANDbox\hskip\wd\ANDbox
        \ignorespaces
    }
    \newsavebox\ANDbox
    \sbox\ANDbox{}

    \begin{header}
        \begin{center}
            \textbf{\Huge \scshape  Блинов Илья} \\ \vspace{1pt}
            \small {{ulkhan.2014@gmail.com}} $|$ Telegram:
            \href{https://t.me/iiblinov}{{@iiblinov}} $|$
            \href{https://github.com/Fraudik}{{github.com/Fraudik}}
        \end{center}
    
        % \normalsize
        % % \mbox{{\color{black}\footnotesize\faMapMarker*}\hspace*{0.13cm}Tula, Russian Federation}%
        % % \kern 0.25 cm%
        % % \AND%
        % % \kern 0.25 cm%
        % \mbox{{\color{black}{\footnotesize\faEnvelope[regular]}\hspace*{0.13cm}	ulkhan.2014@gmail.com}}
        % \kern 0.25 cm%
        % \AND%
        % \kern 0.25  cm%
        % \mbox{Telegram: \hrefWithoutArrow{https://t.me/iiblinov}{\color{blue}{@iiblinov}}}%
        % \kern 0.25 cm%
        % \AND%
        % \kern 0.25 cm%
        % \mbox{\hrefWithoutArrow{https://github.com/Fraudik}{\color{blue}{\footnotesize\faGithub}\hspace*{0.13cm}Fraudik}}%
    \end{header}

    \section{Навыки}

        
        \begin{onecolentry}
            \textbf{Разработка:} Python, SQL, asyncio, gevent, Pandas, Spark, NumPy, SQLAlchemy, FastAPI, Django (ORM, REST framework and Graphene), Flask, C++

        \end{onecolentry}

        \vspace{0.1 cm}

        \begin{onecolentry}
            \textbf{Технологии:} REST, GraphQL, gRPC, RabbitMQ, Kafka, Celery, PostgreSQL, SQLite, Redis

        \end{onecolentry}

        \vspace{0.1 cm}

        \begin{onecolentry}
            \textbf{DevOps \& Cloud:}  nginx, Gunicorn, Docker, Kubernetes (entry-level), Terraform, Ansible, CI \& CD, Git, bash, Grafana, Prometheus, Airflow
        \end{onecolentry}

        \vspace{0.1 cm}

        \begin{onecolentry}
            \textbf{Языки:} Русский (носитель), Английский (C1)
        \end{onecolentry}
        

    \section{Опыт работы}

   
        \begin{twocolentry}{        
        {С Мая 2023}}
            \textbf{PriceCube, Разработчик бэкенда, Middle}
            
            {Сервис ценообразования с помощью машинного обучения на маркетплейсах}
        \end{twocolentry}

        \vspace{0.10 cm}
        \begin{onecolentry}
            \begin{highlights}
                \item Разработал с нуля и интегрировал ETL модули для сбора и обработки данных по API
                \item Разрабатывал продуктовые фичи от этапа проектирования до этапа выкатки в production
                \item Увеличил производительность существующей системы сбора данных более чем в 5 раз \\ за счет асинхронной обработки с помощью потоков, Celery, asyncio и кэширования данных
                \item Значительно ускорил запросы на GraphQL с помощью витрин данных, пагинации и кэширования
                \item Внедрил архитектуру с Kafka для балансировки нагрузки от произвольного количества периодических задач, получив снижение пиковой нагрузки во время их выполнения более чем в 10 раз
                \item Внедрил практику обсуждения User Flow перед разработкой новых продуктовых фич, что позволило сократить время дальнейшей разработки в среднем в 1,5 раза
            \end{highlights}
        \end{onecolentry}

        \vspace{0.2 cm}

        \begin{twocolentry}{
            
        {Сентябрь 2022 – Май 2023}}
        \textbf{Система психологических опросов, Разработчик бэкенда}

        {Микросервисная система для сбора и анализа психологических опросов}
        \end{twocolentry}

        \vspace{0.10 cm}
        \begin{onecolentry}
            \begin{highlights}
                \item Руководил командой из 5 разработчиков над микросервисом
            \end{highlights}
        \end{onecolentry}

        \vspace{0.2 cm}

        \begin{twocolentry}{
            
        {Лето 2024}}
            \textbf{Летняя Олимпиадная школа МФТИ, Лектор}
        \end{twocolentry}

        \vspace{0.10 cm}
        \begin{onecolentry}
            \begin{highlights}
                \item Подготавливал и читал лекции, а также проводил семинары по темам алгоритмов и структур данных
            \end{highlights}
        \end{onecolentry}

        \vspace{0.2 cm}

        \begin{twocolentry}{
            
        {Сентябрь 2022 – Июнь 2025}}
            \textbf{ВШЭ, Учебный ассистент}
        \end{twocolentry}

        \vspace{0.10 cm}
        \begin{onecolentry}
            \begin{highlights}
                \item Ассистировал на курсах «Программирование на Python», «Программирование на C++», \\ «Алгоритмы и структуры данных»
                \item Руководил учебными ассистентами на курсах «Программирование на C++», \\ «Алгоритмы и структуры данных»
            \end{highlights}
        \end{onecolentry}
        

    \section{Образование}

        \begin{twocolentryED}{
            
            
        {2021 – 2025}}
            \textbf{ВШЭ, ФКН, Прикладная математика и информатика}
        \end{twocolentryED}

        \vspace{0.10 cm}
        \begin{onecolentry}
            \begin{highlights}
                \item \textbf{GPA}: \underline{9.6}
                \item \textbf{Релевантные курсы:}
                Распределенные системы, Дизайн систем,
                Системы баз данных, Облачные вычисления, \\ Теория и практика многопоточной синхронизации,
                Архитектура компьютеров и операционные системы, Компьютерные сети, Python, C++
                \item \textbf{Minor}: UX-дизайн
            \end{highlights}
        \end{onecolentry}

        \vspace{0.2 cm}

        \begin{twocolentryED}{
            
            
        {2018 – 2020}}
            \textbf{Яндекс Лицей, Промышленное программирование на Python}
        \end{twocolentryED}

        \vspace{0.10 cm}
        \begin{onecolentry}
            \begin{highlights}
                \item \textbf{Вошёл в топ выпускников проекта}, номер сертификата: 2002 21215 \ (\href{https://lms.yandex.ru/certificate/check}{ссылка для проверки})
                \item \textbf{Релевантные курсы:}  
                Веб-разработка, Работа с базами данных, Асинхронное программирование, Python
            \end{highlights}
        \end{onecolentry}


    \section{Достижения}

        
        \begin{samepage}
            \begin{twocolentry}{
                2023
            }
                \mbox{\textbf{Лауреат стипендии имени Ильи Сегаловича} (топ 10/3000+ студентов ФКН) \ \href{https://cs.hse.ru/iseg}{ссылка}}
                %\\ \mbox{(Топ 10/3000+ студентов ФКН)}
            \end{twocolentry}

        % \vspace{0.1 cm}

             \begin{twocolentry}{
                2024
            }
                \mbox{\textbf{Победитель конкурса Яндекса ассистентов преподавателей IT-дисциплин} \ \href{https://education.yandex.ru/teaching-assistants}{ссылка}} 
                % \\ \mbox{(со всей страны)}
            \end{twocolentry}

             \begin{twocolentry}{
                Второй семестр, 2024
            }
                \mbox{\textbf{Лауреат стипендии ФКН ВШЭ \ "Лучший учебный ассистент"} \ \href{https://cs.hse.ru/initiative/bestassistants}{ссылка}}
            \end{twocolentry}
            
        \end{samepage}

                \vspace{-0.2 cm}

    \section{Исследования}
        
        \begin{samepage}
            \begin{twocolentryED}{
                2025 (wip)
            }
                \mbox{И.Блинов, \textbf{Параллельная симуляция поведения мультиагентных систем с неблокирующей}} \\ \mbox{\textbf{синхронизацией, моделируемых базовыми сетями Петри и сетями рабочих процессов}}
            \end{twocolentryED}
        \end{samepage}


                \vspace{-0.2 cm}

       \section{Проекты}

        \begin{twocolentryED}{
            
        {2024}}
            \textbf{Большие простые числа, Курсовой проект}
        \end{twocolentryED}

        \vspace{0.10 cm}
        \begin{onecolentry}
            \begin{highlights}
                \item Продвинутые алгоритмы проверки чисел на простоту на C++ с использованием GMPXX
                (\href{https://github.com/Fraudik/LargePrimeNumbers}{ссылка})
            \end{highlights}
        \end{onecolentry}

        \vspace{0.2 cm}

        \begin{twocolentryED}{

        {2022}}
            \textbf{Обработчик изображений BMP, Учебный проект}
        \end{twocolentryED}

        \vspace{0.10 cm}
        \begin{onecolentry}
            \begin{highlights}
                \item Консольное приложение на C++ для преобразования изображений в формате BMP
                (\href{https://github.com/Fraudik/BMP-Image-Processor}{ссылка})
            \end{highlights}
        \end{onecolentry}

\end{document}
